\documentclass[]{article}
\usepackage{lmodern}
\usepackage{amssymb,amsmath}
\usepackage{ifxetex,ifluatex}
\usepackage{fixltx2e} % provides \textsubscript
\ifnum 0\ifxetex 1\fi\ifluatex 1\fi=0 % if pdftex
  \usepackage[T1]{fontenc}
  \usepackage[utf8]{inputenc}
\else % if luatex or xelatex
  \ifxetex
    \usepackage{mathspec}
  \else
    \usepackage{fontspec}
  \fi
  \defaultfontfeatures{Ligatures=TeX,Scale=MatchLowercase}
\fi
% use upquote if available, for straight quotes in verbatim environments
\IfFileExists{upquote.sty}{\usepackage{upquote}}{}
% use microtype if available
\IfFileExists{microtype.sty}{%
\usepackage{microtype}
\UseMicrotypeSet[protrusion]{basicmath} % disable protrusion for tt fonts
}{}
\usepackage[margin=1in]{geometry}
\usepackage{hyperref}
\hypersetup{unicode=true,
            pdftitle={performance},
            pdfauthor={Murugesan Nagarajan},
            pdfborder={0 0 0},
            breaklinks=true}
\urlstyle{same}  % don't use monospace font for urls
\usepackage{color}
\usepackage{fancyvrb}
\newcommand{\VerbBar}{|}
\newcommand{\VERB}{\Verb[commandchars=\\\{\}]}
\DefineVerbatimEnvironment{Highlighting}{Verbatim}{commandchars=\\\{\}}
% Add ',fontsize=\small' for more characters per line
\usepackage{framed}
\definecolor{shadecolor}{RGB}{248,248,248}
\newenvironment{Shaded}{\begin{snugshade}}{\end{snugshade}}
\newcommand{\KeywordTok}[1]{\textcolor[rgb]{0.13,0.29,0.53}{\textbf{#1}}}
\newcommand{\DataTypeTok}[1]{\textcolor[rgb]{0.13,0.29,0.53}{#1}}
\newcommand{\DecValTok}[1]{\textcolor[rgb]{0.00,0.00,0.81}{#1}}
\newcommand{\BaseNTok}[1]{\textcolor[rgb]{0.00,0.00,0.81}{#1}}
\newcommand{\FloatTok}[1]{\textcolor[rgb]{0.00,0.00,0.81}{#1}}
\newcommand{\ConstantTok}[1]{\textcolor[rgb]{0.00,0.00,0.00}{#1}}
\newcommand{\CharTok}[1]{\textcolor[rgb]{0.31,0.60,0.02}{#1}}
\newcommand{\SpecialCharTok}[1]{\textcolor[rgb]{0.00,0.00,0.00}{#1}}
\newcommand{\StringTok}[1]{\textcolor[rgb]{0.31,0.60,0.02}{#1}}
\newcommand{\VerbatimStringTok}[1]{\textcolor[rgb]{0.31,0.60,0.02}{#1}}
\newcommand{\SpecialStringTok}[1]{\textcolor[rgb]{0.31,0.60,0.02}{#1}}
\newcommand{\ImportTok}[1]{#1}
\newcommand{\CommentTok}[1]{\textcolor[rgb]{0.56,0.35,0.01}{\textit{#1}}}
\newcommand{\DocumentationTok}[1]{\textcolor[rgb]{0.56,0.35,0.01}{\textbf{\textit{#1}}}}
\newcommand{\AnnotationTok}[1]{\textcolor[rgb]{0.56,0.35,0.01}{\textbf{\textit{#1}}}}
\newcommand{\CommentVarTok}[1]{\textcolor[rgb]{0.56,0.35,0.01}{\textbf{\textit{#1}}}}
\newcommand{\OtherTok}[1]{\textcolor[rgb]{0.56,0.35,0.01}{#1}}
\newcommand{\FunctionTok}[1]{\textcolor[rgb]{0.00,0.00,0.00}{#1}}
\newcommand{\VariableTok}[1]{\textcolor[rgb]{0.00,0.00,0.00}{#1}}
\newcommand{\ControlFlowTok}[1]{\textcolor[rgb]{0.13,0.29,0.53}{\textbf{#1}}}
\newcommand{\OperatorTok}[1]{\textcolor[rgb]{0.81,0.36,0.00}{\textbf{#1}}}
\newcommand{\BuiltInTok}[1]{#1}
\newcommand{\ExtensionTok}[1]{#1}
\newcommand{\PreprocessorTok}[1]{\textcolor[rgb]{0.56,0.35,0.01}{\textit{#1}}}
\newcommand{\AttributeTok}[1]{\textcolor[rgb]{0.77,0.63,0.00}{#1}}
\newcommand{\RegionMarkerTok}[1]{#1}
\newcommand{\InformationTok}[1]{\textcolor[rgb]{0.56,0.35,0.01}{\textbf{\textit{#1}}}}
\newcommand{\WarningTok}[1]{\textcolor[rgb]{0.56,0.35,0.01}{\textbf{\textit{#1}}}}
\newcommand{\AlertTok}[1]{\textcolor[rgb]{0.94,0.16,0.16}{#1}}
\newcommand{\ErrorTok}[1]{\textcolor[rgb]{0.64,0.00,0.00}{\textbf{#1}}}
\newcommand{\NormalTok}[1]{#1}
\usepackage{longtable,booktabs}
\usepackage{graphicx,grffile}
\makeatletter
\def\maxwidth{\ifdim\Gin@nat@width>\linewidth\linewidth\else\Gin@nat@width\fi}
\def\maxheight{\ifdim\Gin@nat@height>\textheight\textheight\else\Gin@nat@height\fi}
\makeatother
% Scale images if necessary, so that they will not overflow the page
% margins by default, and it is still possible to overwrite the defaults
% using explicit options in \includegraphics[width, height, ...]{}
\setkeys{Gin}{width=\maxwidth,height=\maxheight,keepaspectratio}
\IfFileExists{parskip.sty}{%
\usepackage{parskip}
}{% else
\setlength{\parindent}{0pt}
\setlength{\parskip}{6pt plus 2pt minus 1pt}
}
\setlength{\emergencystretch}{3em}  % prevent overfull lines
\providecommand{\tightlist}{%
  \setlength{\itemsep}{0pt}\setlength{\parskip}{0pt}}
\setcounter{secnumdepth}{0}
% Redefines (sub)paragraphs to behave more like sections
\ifx\paragraph\undefined\else
\let\oldparagraph\paragraph
\renewcommand{\paragraph}[1]{\oldparagraph{#1}\mbox{}}
\fi
\ifx\subparagraph\undefined\else
\let\oldsubparagraph\subparagraph
\renewcommand{\subparagraph}[1]{\oldsubparagraph{#1}\mbox{}}
\fi

%%% Use protect on footnotes to avoid problems with footnotes in titles
\let\rmarkdownfootnote\footnote%
\def\footnote{\protect\rmarkdownfootnote}

%%% Change title format to be more compact
\usepackage{titling}

% Create subtitle command for use in maketitle
\newcommand{\subtitle}[1]{
  \posttitle{
    \begin{center}\large#1\end{center}
    }
}

\setlength{\droptitle}{-2em}
  \title{performance}
  \pretitle{\vspace{\droptitle}\centering\huge}
  \posttitle{\par}
  \author{Murugesan Nagarajan}
  \preauthor{\centering\large\emph}
  \postauthor{\par}
  \predate{\centering\large\emph}
  \postdate{\par}
  \date{5/8/2018}


\begin{document}
\maketitle

\section{Analysis}\label{analysis}

Installed the caret package and then trying to apply the GBM,Random
Forest, Neural Network and Logistic regression from the caret package.

The below code will read the data from the CSV file

\begin{Shaded}
\begin{Highlighting}[]
\KeywordTok{library}\NormalTok{(}\StringTok{"caret"}\NormalTok{)}

\NormalTok{raw_data <-}\StringTok{ }\KeywordTok{read.csv}\NormalTok{(}\StringTok{'/Projects/R/CSX415-project/phone_mark/data/bank.csv'}\NormalTok{,}\DataTypeTok{sep=}\StringTok{';'}\NormalTok{)}
\end{Highlighting}
\end{Shaded}

\section{Data split}\label{data-split}

Use the caret function to divide the data into two set 75\% of data as
training set and 25\% of data as testset

\begin{Shaded}
\begin{Highlighting}[]
\NormalTok{index<-}\KeywordTok{createDataPartition}\NormalTok{(raw_data}\OperatorTok{$}\NormalTok{y,}\DataTypeTok{p=}\FloatTok{0.5}\NormalTok{,}\DataTypeTok{list=}\OtherTok{FALSE}\NormalTok{)}
\NormalTok{trainset<-}\StringTok{ }\NormalTok{raw_data[index,]}
\NormalTok{testset<-raw_data[}\OperatorTok{-}\NormalTok{index,]}
\end{Highlighting}
\end{Shaded}

\section{Future selection using
caret}\label{future-selection-using-caret}

\begin{Shaded}
\begin{Highlighting}[]
\NormalTok{outcomeName<-}\StringTok{'y'}
\NormalTok{predictors<-}\KeywordTok{names}\NormalTok{(trainset)[}\OperatorTok{!}\KeywordTok{names}\NormalTok{(trainset) }\OperatorTok\StringTok{ }\NormalTok{outcomeName]}
\end{Highlighting}
\end{Shaded}

\section{Training the model}\label{training-the-model}

Apply GBM, Random forest, Neural Net and Logistic Regression

\begin{Shaded}
\begin{Highlighting}[]
\NormalTok{model_gbm<-}\KeywordTok{train}\NormalTok{(trainset[,predictors],trainset[,outcomeName],}\DataTypeTok{method=}\StringTok{'gbm'}\NormalTok{)}
\NormalTok{model_rf<-}\KeywordTok{train}\NormalTok{(trainset[,predictors],trainset[,outcomeName],}\DataTypeTok{method=}\StringTok{'rf'}\NormalTok{)}
\NormalTok{model_nnet<-}\KeywordTok{train}\NormalTok{(trainset[,predictors],trainset[,outcomeName],}\DataTypeTok{method=}\StringTok{'nnet'}\NormalTok{)}
\NormalTok{model_glm<-}\KeywordTok{train}\NormalTok{(trainset[,predictors],trainset[,outcomeName],}\DataTypeTok{method=}\StringTok{'glm'}\NormalTok{)}
\end{Highlighting}
\end{Shaded}

\section{Prediction}\label{prediction}

Predict the outcome of test data and then evaluate the model performance

\subsection{Evaluate the random forest model using the
metrics}\label{evaluate-the-random-forest-model-using-the-metrics}

\begin{Shaded}
\begin{Highlighting}[]
\NormalTok{predict<-}\KeywordTok{predict.train}\NormalTok{(}\DataTypeTok{object=}\NormalTok{model_rf,testset[,predictors],}\DataTypeTok{type=}\StringTok{"raw"}\NormalTok{)}
\KeywordTok{table}\NormalTok{(predict)}
\KeywordTok{confusionMatrix}\NormalTok{(predict,testset[,outcomeName])}
\end{Highlighting}
\end{Shaded}

\section{Model performance for Random
Forest}\label{model-performance-for-random-forest}

\begin{longtable}[]{@{}lll@{}}
\toprule
Prediction & no & yes\tabularnewline
\midrule
\endhead
No & 1927 & 161\tabularnewline
Yes & 73 & 99\tabularnewline
\bottomrule
\end{longtable}

\subsection{evaluate the GBM model and find the
metrics}\label{evaluate-the-gbm-model-and-find-the-metrics}

\begin{Shaded}
\begin{Highlighting}[]
\NormalTok{predict<-}\KeywordTok{predict.train}\NormalTok{(}\DataTypeTok{object=}\NormalTok{model_gbm,testset[,predictors],}\DataTypeTok{type=}\StringTok{"raw"}\NormalTok{)}
\KeywordTok{table}\NormalTok{(predict)}
\KeywordTok{confusionMatrix}\NormalTok{(predict,testset[,outcomeName])}
\end{Highlighting}
\end{Shaded}

\section{Model performance for GBM}\label{model-performance-for-gbm}

\begin{longtable}[]{@{}lll@{}}
\toprule
Prediction & no & yes\tabularnewline
\midrule
\endhead
No & 1954 & 189\tabularnewline
Yes & 46 & 71\tabularnewline
\bottomrule
\end{longtable}

\subsection{evaluate the Neural Net model and calculate the
metrics.}\label{evaluate-the-neural-net-model-and-calculate-the-metrics.}

\begin{Shaded}
\begin{Highlighting}[]
\NormalTok{predict<-}\KeywordTok{predict.train}\NormalTok{(}\DataTypeTok{object=}\NormalTok{model_nnet,testset[,predictors],}\DataTypeTok{type=}\StringTok{"raw"}\NormalTok{)}
\KeywordTok{table}\NormalTok{(predict)}
\KeywordTok{confusionMatrix}\NormalTok{(predict,testset[,outcomeName])}
\end{Highlighting}
\end{Shaded}

\section{Model performance for Neural
Networks}\label{model-performance-for-neural-networks}

\begin{longtable}[]{@{}lll@{}}
\toprule
Prediction & no & yes\tabularnewline
\midrule
\endhead
No & 1930 & 147\tabularnewline
Yes & 70 & 113\tabularnewline
\bottomrule
\end{longtable}

\subsection{evaluate the glm and calculate the
metrics.}\label{evaluate-the-glm-and-calculate-the-metrics.}

\begin{Shaded}
\begin{Highlighting}[]
\NormalTok{predict<-}\KeywordTok{predict.train}\NormalTok{(}\DataTypeTok{object=}\NormalTok{model_glm,testset[,predictors],}\DataTypeTok{type=}\StringTok{"raw"}\NormalTok{)}
\KeywordTok{confusionMatrix}\NormalTok{(predict,testset[,outcomeName])}
\end{Highlighting}
\end{Shaded}

\section{Model performance for Logistic
Regression}\label{model-performance-for-logistic-regression}

\begin{longtable}[]{@{}lll@{}}
\toprule
Prediction & no & yes\tabularnewline
\midrule
\endhead
No & 1962 & 179\tabularnewline
Yes & 38 & 81\tabularnewline
\bottomrule
\end{longtable}

\section{Conclusion}\label{conclusion}

Calculated the accuracy and Sensitivity for all these model and listed
down

\begin{longtable}[]{@{}lll@{}}
\toprule
Model & Accuracy & Sensitivity\tabularnewline
\midrule
\endhead
GBM & 90 & 98\tabularnewline
Random Forest & 89 & 99\tabularnewline
Neural Network & 90/97\tabularnewline
Logistic Regression & 90 & 98\tabularnewline
\bottomrule
\end{longtable}


\end{document}
